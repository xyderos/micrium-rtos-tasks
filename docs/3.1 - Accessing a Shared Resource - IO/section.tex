\setcounter{section}{3}
\setcounter{subsection}{0}

\subsection{Accessing a Shared Resource: I/O}

\subsubsection*{What is the expected output of this program in the first glance?}
The program consists of 3 tasks. The first two simply echo "Hello from Task[number]" where [number] is the task number (1 or 2). The third task outputs information about stack space usage of each task. From looking at the code, an expected output would be:
\begin{verbatim}
Hello from Task1 
Hello from Task2
Task1 
\end{verbatim}

\subsubsection*{The program might not show the desired behavior. Explain why this might happen.}
While not completely unexpected, some output from a task might be mixed with output from a different task due to preemptiion from a task with a higher priority. The highest priority task (Task 1) should however not be preempted since it has the highest priority.

\subsubsection*{What will happen if you take away OSTimeDlyHMSM() statements? Why?}
Task 1 will loop indefinitely and without being preempted by any other task since it has the highest priority. The program will simply output "Hello from Task1" in an infinite loop.

\subsubsection*{Semaphores can help you to get the desired program behavior. What are semaphores? How can one declare and create a semaphore in MicroC/OS-II?}
Semaphores are a low level signal mechanism which provides a way to control access to resources. A integer variable which can be incremented and decremented in a thread-safe fashion is one way to describe a semaphore.
\ucosii has built in functions for creating and managing semaphores. By initializing an \texttt{OS\_EVENT} structure using \texttt{OSSemCreate()}, a semaphore is created which can be used with various functions such as \texttt{OSSemPend()}, \texttt{OSSemPost()} among other (\href{https://doc.micrium.com/pages/viewpage.action?pageId=16879606}{\ucosii API Reference \ExternalLink}).

\subsubsection*{How can a semaphore protect a critical section? Give a code example!}
\begin{minipage}{\linewidth}
\begin{lstlisting}[style=CStyle]
// declare the semaphore as global
OS_EVENT * criticalSemaphore;

// in main thread...
void main() {
    // create and initialize the semaphore with an initial value of 1 (not busy)
    criticalSemaphore = OSSemCreate(1); 

    // ...
}

// in task code:
void task() {
    // wait until criticalSemaphore has a value of 1
    OSSemPend(criticalSemaphore, 0, &err);

    // do critical stuff
    criticalstuff();
    
    // signal the semaphore, indicating that the critical part is done
    // some other task waiting on OSSemPend() can now continue
    OSSemPost(task1StateSemaphore);
}

\end{lstlisting}
\end{minipage}

\subsubsection*{Who is allowed to lock and release semaphores? Can a task release a semaphore that was locked by another task?}
There are no safeguards in \ucosii to prevent unwanted lock and unlocking behaviour of individual tasks, so it's up to the programmer to make sure not to call \texttt{OSSemPost()} without a preceding \texttt{OSSemPend()} for example.

\subsubsection*{Explain the mechanisms behind the command OSSemPost() and OSSemPend()!}
TODO
\subsubsection*{Draw the new application as a block diagram containing processes, semaphores and shared resources. Use the graphical notation which has been used in the lectures and exercises for this purpose, and included in Figure 4 from Appendix A.}
TODO

